\section{Results}

\subsection{Estimating traits via PROSPECT inversion}

\begin{figure}
  \centering
  \includegraphics[width=\textwidth]{{figures/project_validation_summary}.pdf}
  \caption{\
    Validation of PROSPECT against observed trait values, by project and PROSPECT version.
    Y-axis represents $R^2$ values for robust linear regression.
  }\label{fig:project_validation_summary}
\end{figure}

Across most projects and traits, the four different PROSPECT versions performed similarly in terms of their ability to retrieve traits (Figure~\ref{fig:project_validation_summary}).
For all versions of PROSPECT, leaf water content was consistently the most accurate trait retrieved, while retrievals of other traits were highly project-specific.
For several projects spanning a large range of species (Cali.\ Eco.\ Traits, NASA FFT, and NASA HyspIRI), moving from chlorophyll as the only pigment (PROSPECT 4) to chlorophyll and carotenoids (PROSPECT 5/5B) drastically reduced inversion accuracy of dry matter contents, but this accuracy was restored by the further addition of anthocyanins and modification of the refractive index in PROSPECT D (Figure~\ref{fig:project_validation_summary}).
Because PROSPECT-D also retrieves anthocyanin content and generally performed as well or better than other versions, it was the version selected for subsequent analyses.

\begin{figure}
  \centering
  \includegraphics[width=\textwidth]{3_prospect/figures/validation_by_gf.pdf}
  \caption{\
    Validation of PROSPECT-D against observed trait values.
    Grey lines indicate 95\% confidence intervals around trait estimates.
    Solid, colored lines are robust regressions fit to the data by project and functional type.
    The solid black line is a regression fit to all of the data for a given functional type.
    The dashed black line is a 1 to 1 fit.
  }\label{fig:prospect_D_validation}
\end{figure}

\begin{figure}
  \centering
  \includegraphics[width=\textwidth]{3_prospect/figures/r2_by_gf.pdf}
  \caption{%
    Regression-based validation of trait retrieval using Bayesian PROSPECT-D inversion.
    Each bar represents the $R^2$ value of a robust linear regression of observed trait value as a function of inversion estimate for a particular species (including additive and multiplicative bias correction).
    Colors indicate species functional type, as determined by the interaction of growth form and leaf morphology.
  }\label{fig:prospect_D_r2}
\end{figure}

Inversion accuracy varied significantly by project and growth form (Figures~\ref{fig:project_validation_summary},~\ref{fig:prospect_D_validation}, and~\ref{fig:prospect_D_r2}).
In terms of regression $R^2$, inversion accuracy was highest for broadleaved trees, lower for herbs and needleleaved trees, and lowest for grasses.
However, there was substantial project-specific variability in accuracy between these groups.
For example, both water and LMA retrievals from the California Ecosystem Traits dataset were consistently much worse than for other datasets for both broadleaved and needleleaved trees, while the LOPEX and ANGERS datasets (against which PROSPECT is calibrated) performed very well for all traits for broadleaved trees, herbs, and grasses.
The full pairs plot (Figure~\ref{fig:prospect_D_validation}) reveals that regression $R^2$ is insufficient for capturing all of the patterns in the validation.
In several cases (e.g.\ water and LMA retrieval for conifers from the NASA FFT dataset, or carotenoid retrievals from the soybean aphid dataset), there is a saturation effect, whereby accuracy is good at lower trait values but declines as trait values increase.
In other cases, there is a significant additive and/or multiplicative bias in retrievals --- for instance, in the retrieval of chlorophyll and carotenoid contents from herbs.

\subsection{Drivers of variability in leaf optical traits}

\begin{figure}
  \includegraphics[width=\textwidth]{{figures/within_vs_across}.pdf}
  \caption{\
    Fraction of variance in each optical trait explained by species identity,
    based on analysis of variance on least-squares linear regression.
  }\label{fig:within_vs_across}
\end{figure}

\begin{figure}
  \includegraphics[width=\textwidth]{{figures/across_species_anova}.pdf}
  \caption{\
    Fraction of across-species variance in each optical trait (i.e.~species means) explained by species attributes,
    based on analysis of variance on least-squares linear regression.
    A star (*) indicates attribute effects significant at the 90\% confidence interval.
    Attributes are as follows:
    \texttt{myco\_is\_am} --- Mycorrhizal fungi association (arbuscular or other such as ectomycorrhizal or ericoid);
    \texttt{nitrogen\_fixer} --- whether the species is a Nitrogen fixer;
    \texttt{growth\_form} --- tree, shrub, herb, or grass;
    \texttt{ps\_type} --- photosynthetic pathway (C3 or C4);
    \texttt{leaf\_type} --- leaf morphology (broadleaved or needleleaved);
    \texttt{phenology} --- whether the species is deciduous or evergreen.
  }\label{fig:across_species_anova}
\end{figure}

Across all optical traits, roughly half of variability was explained by species identity (Figure~\ref{fig:within_vs_across}).
The variance across species means was largely idiosyncratic to species, with only up to 25\% of variance explainable by species attributes (Figure~\ref{fig:across_species_anova}).
The most important explanatory attribute was leaf phenology (deciduous vs.~evergreen), with occasional significant effects for leaf type (broad vs.~needle), growth form (woody vs.~herbaceous), and mycorrhizal association (arbuscular or non-arbuscular). 

\begin{figure}
  \centering
  \includegraphics[width=\textwidth]{{figures/treatment_summary}.pdf}
  \caption{\
    Effects of different sources of intraspecific variability on traits estimated via PROSPECT inversion.
    Each value is the fixed effect slope on the corresponding trait normalized to zero mean and unit variance, as estimated from a linear fixed-effects model.
    Color brightness indicates degree of statistical significance (90 or 95\% confidence level), and color hues indicate effect direction (positive or negative).
  }\label{fig:treatment_summary}
\end{figure}

Leaf optical traits responded significantly to a range of natural and experimental stressors (Figure~\ref{fig:treatment_summary}).
Across the entire dataset, intraspecific variability in optical traits was weakly but, in some cases, significantly related to climate, with the strongest effects being declines in mesophyll structure and dry matter content with increasing temperature.
Canopy light environment (i.e.\ whether a leaf was sunlit or shaded) also had a significant effect on most traits, at least for species on which a comparison was possible.
Specifically, shaded leaves showed higher chlorophyll and anthocyanin concentrations and reduced mesophyll structure and water and dry matter contents.

Based on leaf reflectance measurements of \textit{Populus deltoides} (eastern cottonwood) at the University of Arizona by Barnes et al.~(2017), \nocite{barnes_2017_beyond}
seasonal variations in vapor pressure deficit---but not leaf temperature---had significant negative effects on all traits, with the strongest effects on chlorophyll and anthocyanin contents.
Similarly, warming and drought experiments on \textit{Asclepias syriaca} (common milkweed)  had strong and significant effects on almost all optical traits, with both treatments leading to significant decreases in leaf water content and effective number of mesophyll layers and increases in pigments and dry matter content concentrations~\cite{milkweed_data}.
Leaf optical traits also responded significantly to chemical and biotic stressors.
Based on data from Di Vittorio (2009) \nocite{divittorio_2009_enhancing}, needles of \textit{Pinus ponderosa} and \textit{Pinus jeffreyi} from the northern Sierra Nevada mountains experienced reductions in all traits, but most strongly in pigment and dry matter contents,
when afflicted with winter fleck (patchy mortality of needle epidermal cells, usually triggered by exposure to harsh winter weather), sucking and scale insect, and especially ozone damage.
As well, spectral inversion revealed small but statistically significant declines across all optical traits except anthocyanins in \textit{Solanum tuberosum} (potato) plants infected with potato virus Y~\cite{solanum_pvy_data}.
On the other hand, treatment of \textit{Glycine max} (soybean) with aphids resulted in a small but significant increase in pigment concentrations, with the strongest effect observed at medium-level treatment~\cite{soybean_data}.

\begin{figure}
  \includegraphics[width=\textwidth]{{figures/trait_phenology}.pdf}
  \caption{\
    Optical trait estimates through a season for \textit{Quercus rubra} (red oak) at Martha's Vineyard, MA by Yang et al.~(2016).
    Colors indicate sunlit vs.~shaded leaves.
    Line is a LOESS best fit with shaded standard error.
  }\label{fig:trait_phenology}
  \nocite{yang_2016_seasonal}
\end{figure}

Where such measurements were available, leaf optical traits exhibited a strong phenological signal (Figure~\ref{fig:trait_phenology}).
All optical traits showed a peak in late July / early August, followed by a decline into the fall, with the sharpest declines for pigments and water content and less precipitous declines for dry matter and mesophyll structure.
Furthermore, the effective number of leaf mesophyll layers, and to a lesser extent, leaf dry matter content in shade leaves, appeared to increase in the late fall.
With the exception of anthocyanin content, all traits for shaded leaves were higher and experienced a greater seasonal variability than sunlit leaves.

\subsection{Trait correlations}

\begin{figure}
  \includegraphics[width=\textwidth]{figures/prospect_pca.pdf}
  \caption{\
    Principal components analysis of optical trait correlation matrix. 
    (Left) Cumulative fraction of variance explained by each principal component.
    (Center, Right) Principal component scores and vectors for each optical trait for components 1 and 2 (center) and 2 and 3 (right).
  }\label{fig:prospect_pca}
\end{figure}

Optical traits estimated by PROSPECT are not mutually independent, but rather have some structure to their covariance (Figure~\ref{fig:prospect_pca}).
The first principal component, which explains roughly 50\% of the variability, is defined by increases in all leaf traits and can be interpreted as overall leaf size and tissue density.
The second principal component, which explains an additional 20\% of the variability, is characterized by an approximate trade-off between structural traits (mesophyll structure and dry matter content) and physiological traits (chlorophyll, carotenoid, and water contents).
The third principal component, which explains a further 15\% of the variability, is dominated by a trade-off in water and anthocyanin concentrations.

\begin{figure}
  \centering
  \includegraphics[width=\textwidth]{3_prospect/figures/trait_correlations_lollipop.pdf}
  \caption{\
    Intra-specific pairwise correlations of optical traits with direct measurements of area-based
    leaf N, C, cellulose, lignin, $V_{c,\max}$, and $J_{\max}$.
    Colors indicate plant functional type.
    Species are displayed along the y axis, and are sorted within each facet from highest to lowest average correlation across all traits.
    Analysis was performed only for species with at least 10 pairwise observations of each trait.
  }\label{fig:trait_correlations}
\end{figure}

Covariance of optical traits with six area-normalized traits---leaf nitrogen, carbon, cellulose, and lignin contents, $V_{c,\max}$, and $J_{\max}$---was strongly species specific, but, in many cases, significant (Figure~\ref{fig:trait_correlations}).
In general, for any given species, most of the estimated optical traits exhibited similar correlations with the directly measured trait.
For instance, for Glycine (\textit{Glycine max}, GLMA4), all optical traits were positively correlated with leaf N, C, cellulose, and lignin, whereas for milkweed (\textit{Asclepias syriaca}, ASSY), all of these correlations were negligible.
Leaf nitrogen correlated best for the largest number of species with leaf chlorophyll and, to a slightly lesser extent, with dry matter content.
Leaf carbon and lignin were most consistently correlated with leaf dry matter content, while correlations with cellulose were more idiosyncratic.
$V_{c,\max}$ and $J_{\max}$ were strongly positively correlated with all traits for \textit{Populus deltoides} (PODE3), but completely uncorrelated for milkweed (ASSY).
Correlations between optical and other traits were generally strongest for broadleaf trees, somewhat weaker for needleleaved trees, and weakest for herbs and grasses.
Although this was also the general pattern in the validation, exploratory analyses (not shown) do not find any consistent relationship between trait retrieval accuracy (represented by validation $R^2$) and intraspecific correlation.

\begin{figure}
  \centering
  \includegraphics[width=\textwidth]{3_prospect/figures/trait_correlations_species.pdf}
  \caption{%
    Pairwise correlations among species means for PROSPECT inversion estimates and other, directly-measured area-normalized traits.
    In lower diagonal, circle color and size indicate correlation direction and strength, respectively.
    Values in upper diagonal are correlation coefficients, colored by direction and strength.
  }\label{fig:trait_correlations_species}
\end{figure}

Among species means, almost all area-based traits were at least weakly positively correlated with each other (Figure~\ref{fig:trait_correlations_species}).
Among optical traits, the strongest correlations were among the three pigments, and of leaf water and dry matter contents.
Spectrally-estimated leaf mesophyll structure and dry matter content correlated strongly with traits related to structure, namely C, cellulose, and lignin contents.
Leaf N was most strongly correlated with dry matter and water contents and leaf mesophyll structure, and only weakly correlated with chlorophyll and carotenoid contents.
