\addcontentsline{toc}{chapter}{Curriculum Vitae}

\thispagestyle{empty}

\begin{center}
{\LARGE {\bf CURRICULUM VITAE}}\\
\vspace{0.5in}
{\large {\bf Alexey N Shiklomanov}}
\end{center}

\section*{Education}

\begin{itemize}
  \item Ph.D. Geography, Boston University (Exp. June 2018)
  \item Honors B.S. with Distinction, Chemistry \& Environmental Science, University of Delaware (May 2014)
    \begin{itemize}
      \item \textit{Magna cum laude}
      \item Geography minor
      \item Honors thesis: \textit{Stemflow acid neutralization capacity in a deciduous forest: the role of edge effects.} Advisor: Delphis Levia
    \end{itemize}
\end{itemize}

\section*{Publications}

\subsection*{Published}

\begin{itemize}
  \item 2017.~Kaverin, D.A.; Melnichuk, E.B.; Shiklomanov, N.I.; Kakunov, N.B.; Pastukhov, A.V.; \textbf{Shiklomanov, A.N.}.
    Long-term changes in the ground thermal regime of an artificially-drained thaw lake basin: A case study in the Russian European North.
    \textit{Permafrost and Periglacial Processes}. \textbf{29}(1):49--59.

  \item 2016.~\textbf{Shiklomanov, A.N.}; Dietze, M.C.; Viskari, T.; Townsend, P.A.; Serbin, S.P. 
    Quantifying the influences of spectral resolution on uncertainty in leaf trait estimates through a Bayesian approach to RTM inversion. 
    \textit{Remote Sensing of Environment} \textbf{183}:226--238. 

  \item 2015.~Levia, D.F.; \textbf{Shiklomanov, A.N.}; Van Stan, J.T.; Sheick, C.E.; Inamdar, S.P.; Mitchell, M.J.; McHale, P.J. 
    Calcium and aluminum cycling in a temperate broadleaved deciduous forest of the eastern USA\@: relative impacts of tree species, canopy state, and flux type. 
    \textit{Environmental Monitoring and Assessment} \textbf{187}(7):4675. 

  \item 2014.~\textbf{Shiklomanov, A.N.} \& Levia, D.F. 
    Stemflow acid neutralization capacity in a broadleaved deciduous forest: The role of edge effects. 
    \textit{Environmental Pollution} \textbf{193}:45--53. 
\end{itemize}

\subsection*{In review or revision}

\begin{itemize}
  \item \textbf{Shiklomanov, A.N.}; Cowdery, E.M.; Bahn, M.; Byun, C.; Craine, J.; Gonzalez-Melo, A.; Jansen, S.; Kraft, N.; Kramer, K.; Minden, V.; Niinemets, Ü.; Onoda, Y.; Sosinski, E.; Soudzilovskaia, N.; Dietze, M.C. 
    Does the leaf economic spectrum hold within plant functional types? A Bayesian multivariate trait meta-analysis.
    \textit{New Phytologist}. In revision.

  \item \textbf{Shiklomanov, A.N.}; Smallman, L.; Bradley, B.; Dahlin, K.; Fox, A.; Gough, C.; Hoffman, F.M.; Middleton, E.; Serbin, S.; Smith, W. The role of remote sensing in global change experiments. \textit{Frontiers in Ecology \& Evolution}. In revision.

  \item Viskari, T.; \textbf{Shiklomanov, A.N}.; Dietze, M.C.; Serbin, S.P. The influence of canopy radiation parameter uncertainty on model projections of carbon and energy cycling.
    \textit{Journal of Advances in Modeling Earth Systems}. In review.
\end{itemize}

\section*{Conference \& Workshop Presentations}

\subsection*{Invited}

\begin{itemize}
  \item Ecological Society of America (ESA) Annual Meeting.
  Invited ignite presentation: ``Applications of radiative transfer modeling to vegetation remote sensing''.
  August 2017

  \item American Geophysical Union (AGU) Fall Meeting.
  Invited oral presentation: ``Leaf optical properties shed light on foliar trait variability at individual to global scales''.
  December 2016.
\end{itemize}

\subsection{Contributed}

\begin{itemize}
  \item NASA Biodiversity \& Ecological Forecasting Team Meeting
    Poster presentation: ``Cutting out the middle man: Calibrating and validating an ecosystem model with remotely sensed surface reflectance''.
    April 2018.

  \item American Geophysical Union (AGU) Fall Meeting.
    Oral presentation: ``Leaf optical properties shed light on foliar trait variability at individual to global scales''.
    December 2017.

  \item 39th New Phytologist Symposium --- Trait covariation: Structural and functional relationships in plant ecology.
    Poster presentation: ``Leaf optical properties shed light on foliar trait variability at individual to global scales''.
    June 2017.

  \item NASA Biodiversity \& Ecological Forecasting Team Meeting
    Poster presentation: ``Leaf optical properties shed light on foliar trait variability at individual to global scales''.
    May 2017.

  \item American Association of Geographers (AAG) Fall Meeting.
    Oral presentation: ``Leaf optical properties shed light on foliar trait variability at individual to global scales''.
    April 2017.

  \item American Geophysical Union (AGU) Fall Meeting.
    Oral presentation: ``Scaling and breaking the leaf economic spectrum: Leaf trait relationships diverge across plant functional types''.
    December 2016.

  \item American Geophysical Union (AGU) Fall Meeting. 
    Poster presentation: ``Applications of spectral inversion to understanding vegetation functional trait relationships.''
    December 2015.

  \item NASA Carbon Cycle \& Ecosystems Conference. 
    Poster presentation: ``Chloroplasts to canopies: Analysis of leaf spectral trait variability across spatial scales.'' 
    April 2015.

  \item North American Carbon Program (NACP) Fall Meeting. 
    Poster presentation: ``Characterization of uncertainties in leaf traits through a Bayesian inversion of the PROSPECT model.'' 
    January 2015.
\end{itemize}


\section*{Workshop and working group participation}

\begin{itemize}
  \item USGS Biodiversity and Climate Modeling Workshop Series.
    June 2017, March 2018.

  \item Oak Ridge National Lab Data Distributed Active Archive Center (DAAC) User Working Group (UWG) meeting. 
    May 2017.

  \item INTERFACE Workshop on Frontiers in terrestrial climate feedbacks: Integrating models and experiments to explore climate feedbacks in a managed and warming world. 
    February 2016.

  \item Keck Institute for Space Sciences (KISS) Workshop: Exploring New Multi-Instrument Approaches to Observing Terrestrial Ecosystems and the Carbon Cycle from Space.  
    October 2015.
\end{itemize}


\section*{Teaching experience}

Certified Software Carpentry instructor. Taught the following workshops:
\begin{itemize}
  \item R, Unix Shell, Git --- Federal Reserve Board (August 2017)
  \item Python, Unix Shell, Git, SQL --- University of West Virginia (May 2017)
  \item Python, Unix Shell, Git --- University of California-San Francisco Medical School (December 2016)
\end{itemize}

\section*{Research Grants}

\begin{itemize}
  \item Spring 2016: NASA Earth \& Space Science Fellowship: ``Tracking successional dynamics of foliar traits using remote sensing'', \$30,000
  \item Spring 2015: Boston University Biogeoscience Student Award, \$500
  \item Winter 2014: Senior Thesis Winter Session Scholars Award, \$600
  \item Summer 2013: Delaware Water Resources Center (DWRC) summer research grant: ``Acid neutralization in a deciduous forest: The role of edge effects'', \$3500
  \item Spring 2013: University of Delaware undergraduate research program supply and expense grant, \$700
\end{itemize}


\section*{Awards and Recognition}

\begin{itemize}
  \item Honorable Mention, National Science Foundation Graduate Research Fellowship Program (GRFP), (Spring 2016)
  \item Honorable Mention, National Science Foundation Graduate Research Fellowship Program (GRFP), (Spring 2015)
  \item Outstanding Senior in Environmental Science, University of Delaware (Spring 2014)
  \item Phi Beta Kappa, Alpha of Delaware Chapter (Spring 2014)
  \item Special Merit Award for Outstanding Achievement in Environmental Science, University of Delaware (Spring 2013)
  \item University of Delaware Dean's List (Fall 2010--Spring 2014)
  \item University of Delaware Governor's Scholarship (Spring 2011--Spring 2014)
  \item University of Delaware Scholar --- merit scholarship (Spring 2010--Spring 2014)
\end{itemize}
