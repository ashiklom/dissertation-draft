% ABSTRACT

Vegetation is a critical component of the biosphere that mediates climate, modifies the hydrologic cycle, is a haven for biodiversity, and provides many natural resources.
Unfortunately, current predictions about the future of the terrestrial biosphere are highly uncertain, as evidenced by the huge disparities in model projections of the magnitude and direction of the land carbon sink.
The sluggish pace of progress in ecosystem model improvement is surprising given the explosion of novel data streams over the last two decades, such as the TRY global plant traits database, the opening of the Landsat archive, and the formation of global networks of eddy covariance measurements.
One may conclude that these data streams are not being utilized to their full potential by the dynamic vegetation modeling community.
The overarching objectives of my research are to identify how these various data streams can be used, both independently and in combination, to add constraint on model predictions.

In my first chapter, I explore the potential of the TRY plant traits database for constraining model parameters.
Specifically, I investigate the additional information content in trait correlations and explore the dependence of these correlations on ecological scale.
However, databases like TRY have major spatial gaps and cannot provide information about changes in traits through time.
Estimation of traits from spectral observations, at scales from the leaf to the landscape, is a promising solution to these issues.
In my second chapter, I introduce a method for estimating leaf traits from leaf reflectance spectra based on inversion of a physically-based leaf radiative transfer model, and use this approach to investigate the relative importance of spectral resolution for accurate and precise trait estimates.
In my third chapter, I apply this approach to a large database of leaf spectra spanning hundreds of species from a variety of biomes to quantify drivers of leaf trait variability and patterns of trait covariance at different scales.
Finally, in my last chapter, I synthesize the techniques and results of the previous chapters to calibrate a vegetation model and then investigate the model's ability to reproduce remotely-sensed surface reflectance.
