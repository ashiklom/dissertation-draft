% This file contains all the necessary setup and commands to create
% the preliminary pages according to the buthesis.sty option.

\title{Towards a more educated guessing: Improving ecological forecasts using data constraints}

\author{Alexey N. Shiklomanov}

% Type of document prepared for this degree:
%   1 = Master of Science thesis,
%   2 = Doctor of Philisophy dissertation.
%   3 = Master of Science thesis and Doctor of Philisophy dissertation.
\degree=2

\prevdegrees{B.S., University of Delaware, 2014}

\department{Department of Earth \& Environment}

% Degree year is the year the diploma is expected, and defense year is
% the year the dissertation is written up and defended. Often, these
% will be the same, except for January graduation, when your defense
% will be in the fall of year X, and your graduation will be in
% January of year X+1
\defenseyear{2018}
\degreeyear{2018}

% For each reader, specify appropriate label {First, Second, Third},
% then name, and title. IMPORTANT: The title should be:
%   "Professor of Electrical and Computer Engineering",
% or similar, but it MUST NOT be:
%   Professor, Department of Electrical and Computer Engineering"
% or you will be asked to reprint and get new signatures.
% Warning: If you have more than five readers you are out of luck,
% because it will overflow to a new page. You may try to put part of
% the title in with the name.
\reader{First}{Michael Dietze, PhD}{Associate Professor of Earth \& Environment}
\reader{Second}{Shawn Serbin, PhD}{Associate Ecologist}
\reader{Third}{Mark Friedl}{Professor of Earth \& Environment}
\reader{Fourth}{Diane Thompson}{Assistant Professor of Earth \& Environment}

% Pretty sure Curtis doesn't have to sign...
%\reader{Fifth}{Curtis Woodcock}{Professor of Earth \& Environment}

% The Major Professor is the same as the first reader, but must be
% specified again for the abstract page. Up to 4 Major Professors
% (advisors) can be defined. 
\numadvisors=1
\majorprof{Michael Dietze, PhD}{{Associate Professor of Earth \& Environment}}

%%%%%%%%%%%%%%%%%%%%%%%%%%%%%%%%%%%%%%%%%%%%%%%%%%%%%%%%%%%%%%%%  

%                       PRELIMINARY PAGES
% According to the BU guide the preliminary pages consist of:
% title, copyright (optional), approval,  acknowledgments (opt.),
% abstract, preface (opt.), Table of contents, List of tables (if
% any), List of illustrations (if any). The \tableofcontents,
% \listoffigures, and \listoftables commands can be used in the
% appropriate places. For other things like preface, do it manually
% with something like \newpage\section*{Preface}.

% This is an additional page to print a boxed-in title, author name and
% degree statement so that they are visible through the opening in BU
% covers used for reports. This makes a nicely bound copy. Uncomment only
% if you are printing a hardcopy for such covers. Leave commented out
% when producing PDF for library submission.
%\buecethesistitleboxpage

% Make the titlepage based on the above information.  If you need
% something special and can't use the standard form, you can specify
% the exact text of the titlepage yourself.  Put it in a titlepage
% environment and leave blank lines where you want vertical space.
% The spaces will be adjusted to fill the entire page.
\maketitle
\cleardoublepage

% The copyright page is blank except for the notice at the bottom. You
% must provide your name in capitals.
\copyrightpage
\cleardoublepage

% Now include the approval page based on the readers information
\approvalpage
\cleardoublepage

% Here goes your favorite quote. This page is optional.
\newpage
%\thispagestyle{empty}
\phantom{.}
\vspace{4in}

\begin{singlespace}
\begin{quote}
  Gentlemen, we have run out of money. Now, we must think.
  \begin{flushright}
    ---Winston Churchill
  \end{flushright}
\end{quote}
\end{singlespace}

\cleardoublepage

% The acknowledgment page should go here. Use something like
% \newpage\section*{Acknowledgments} followed by your text.
\newpage
\section*{\centerline{Acknowledgments}}
This dissertation was truly a collaborative effort, and would not have come together without the support of a great number of people.

First, I would like to acknowledge my funding sources---the Boston University Department of Earth \& Environment and NASA---which allowed me to conduct my research freely and without interruption throughout my entire PhD.
This work also relied heavily on supercomupting resources provided to me free of charge by Boston University.

I would also like to thank all of my fellow graduate students at Boston University, both in Earth \& Environment and in Biology,
and particularly those in my year---Andrew Trlica, Paulo Arevalo, and Radost Stanimirova---for your intellectual and emotional support throughout my dissertation.
Thank you, too, to Amy Zanne, Amy Milo, and Marissa Lee for adopting me into your lab when I moved to the Washington, DC area,
and for providing me with extensive feedback and support.
And of course, a particularly warm thank you to past and present members of the Ecological Forecasting lab for your significant intellectual input, and for making my day-to-day routine better. 

A sincere thank you goes out to my dissertation readers---Diane Thompson, Mark Friedl, and Curtis Woodcock---for your helpful feedback on this dissertation at its various stages.
I owe a particularly large debt of gratitude to Shawn Serbin for not only providing the inspiration for a lot of this PhD research, but also for providing openly and unconditionally most of the data for the later chapters and for always being available to answer my questions and provide input on results.
I am also forever grateful to my advisor Michael Dietze for fundamentally revolutionizing the way I think about and perform the scientific process, and for expertly guiding me throughout every step of the PhD process.

Finally, I wanted to thank my family for challenging and supporting me throughout my entire life, and to my fiancee Sarah Weiskopf for basically everything good that has happened to me in the last 7 years.

\vskip 1in

\noindent
Alexey Shiklomanov\\
PhD Candidate\\
E\&E Department

\cleardoublepage

% The abstractpage environment sets up everything on the page except
% the text itself.  The title and other header material are put at the
% top of the page, and the supervisors are listed at the bottom.  A
% new page is begun both before and after.  Of course, an abstract may
% be more than one page itself.  If you need more control over the
% format of the page, you can use the abstract environment, which puts
% the word "Abstract" at the beginning and single spaces its text.

\begin{abstractpage}
% ABSTRACT

Vegetation is a critical component of the biosphere that mediates climate, modifies the hydrologic cycle, is a haven for biodiversity, and provides many natural resources.
Unfortunately, current predictions about the future of the terrestrial biosphere are highly uncertain, as evidenced by the huge disparities in model projections of the magnitude and direction of the land carbon sink.
The sluggish pace of progress in ecosystem model improvement is surprising given the explosion of novel data streams over the last two decades, such as the TRY global plant traits database, the opening of the Landsat archive, and the formation of global networks of eddy covariance measurements.
One may conclude that these data streams are not being utilized to their full potential by the dynamic vegetation modeling community.
The overarching objectives of my research are to identify how these various data streams can be used, both independently and in combination, to add constraint on model predictions.

In my first chapter, I explore the potential of the TRY plant traits database for constraining model parameters.
Specifically, I investigate the additional information content in trait correlations and explore the dependence of these correlations on ecological scale.
However, databases like TRY have major spatial gaps and cannot provide information about changes in traits through time.
Estimation of traits from spectral observations, at scales from the leaf to the landscape, is a promising solution to these issues.
In my second chapter, I introduce a method for estimating leaf traits from leaf reflectance spectra based on inversion of a physically-based leaf radiative transfer model, and use this approach to investigate the relative importance of spectral resolution for accurate and precise trait estimates.
In my third chapter, I apply this approach to a large database of leaf spectra spanning hundreds of species from a variety of biomes to quantify drivers of leaf trait variability and patterns of trait covariance at different scales.
Finally, in my last chapter, I synthesize the techniques and results of the previous chapters to calibrate a vegetation model and then investigate the model's ability to reproduce remotely-sensed surface reflectance.

\end{abstractpage}
\cleardoublepage

% Now you can include a preface. Again, use something like
% \newpage\section*{Preface} followed by your text

% Table of contents comes after preface
\tableofcontents
\cleardoublepage

% If you do not have tables, comment out the following lines
\newpage
\listoftables
\cleardoublepage

% If you have figures, uncomment the following line
\newpage
\listoffigures
\cleardoublepage

% List of Abbrevs is NOT optional (Martha Wellman likes all abbrevs listed)
\chapter*{List of Abbreviations}
\begin{center}
  \begin{tabular}{lll}
    \hspace*{2em} & \hspace*{1in} & \hspace*{4.5in} \\
    PFT & \dotfill & Plant functional type \\
    $V_{c,max}$ & \dotfill & Maximum photosynthetic carboxylation rate \\
    $J_{max}$ & \dotfill & Maximum photosynthetic electron transport rate \\
    $V_{c,max,mass}$ & \dotfill & Maximum photosynthetic carboxylation rate on a leaf mass basis \\
    $J_{max,mass}$ & \dotfill & Maximum photosynthetic electron transport rate on a leaf mass basis \\
    $V_{c,max,area}$ & \dotfill & Maximum photosynthetic carboxylation rate on a leaf area basis \\
    $J_{max,area}$ & \dotfill & Maximum photosynthetic electron transport rate on a leaf area basis \\
    SLA & \dotfill & Specific leaf area \\
    LMA & \dotfill & Leaf mass per unit area \\
    $N_{mass}$ & \dotfill & Leaf nitrogen content per unit mass \\
    $P_{mass}$ & \dotfill & Leaf phosphorus content per unit mass \\
    $Rd_{mass}$ & \dotfill & Leaf dark respiration on a leaf mass basis \\
    $N_{area}$ & \dotfill & Leaf nitrogen content per unit area \\
    $P_{area}$ & \dotfill & Leaf phosphorus content per unit area \\
    $Rd_{area}$ & \dotfill & Leaf dark respiration on a leaf area basis \\
    RTM & \dotfill & Radiative transfer model \\
    PLSR & \dotfill & Partial Least-Squares Regression \\
    AVIRIS & \dotfill & Airborne visible / infrared imaging spectrometer \\
    MCMC & \dotfill & Markov-Chain Monte Carlo \\
    FFT & \dotfill & NASA Forest Functional Types field campaign \\
    PROSPECT & \dotfill & PROSPECT leaf radiative transfer model \\
    ED & \dotfill & Ecosystem Demography (model) \\
    EDR & \dotfill & Coupled PROSPECT-ED leaf-canopy radiative transfer model \\
    DBH & \dotfill & Diameter at breast height \\
    RMSE & \dotfill & Root mean square error \\
  \end{tabular}
\end{center}
\cleardoublepage

% END OF THE PRELIMINARY PAGES

\newpage
\endofprelim
