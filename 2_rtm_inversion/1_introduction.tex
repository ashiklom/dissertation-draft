\section{Introduction}\label{sec:pecanrtm-intro}

The terrestrial biosphere is fundamentally dependent on the interactions between plants and solar radiation through photosynthesis.
Consequently, we can learn a lot about the structure and functioning of ecosystems by studying these interactions in detail, and over the last several decades our capability do so has expanded dramatically.
Specifically, global scale remote sensing observations from satellites such as Landsat, MODIS, and AVHRR have been used to map and monitor vegetation productivity, distribution, and abundance at high temporal frequency~\cite{loveland_2000_development,friedl_2002_global,hansen_2010_quantification,houborg_2015_advances}.
At the landscape scale, satellite and sub-orbital (airborne) platforms with high spatial (e.g. WorldView, < 1 m) and/or spectral (e.g. AVIRIS Classic, 10 nm) resolution sensors have been able to quantify the spatial distribution of canopy structure, nutrient status, and species composition~\cite{asner_2015_quantifying,banskota_2015_inversion,singh_imaging_2015}.
In addition, field spectrometers with the highest available spectral resolution have provided a fast and relatively simple method for characterizing and monitoring leaf physiology, biochemistry, and morphology~\cite{serbin_2012_spectroscopic,couture_2013_spectroscopic,sullivan_2013_foliar,serbin_spectroscopic_2014,zhao_2014_early}.

An important caveat of using spectral information to study vegetation is that the optical properties being measured are often not of primary interest.
Rather, we are interested in physiologically or ecologically meaningful variables such as total biomass, photosynthetic efficiency, species composition, biomass, or biochemistry that drive observed spectral signatures of vegetation and which can be inferred from the optical properties.
This connection is usually made empirically, either by simple regression with spectral vegetation indices (SVIs)~\cite{fassnacht_2015_nondestructive,haboudane_2002_integrated,huete_2002_overview}
or through more advanced statistical methods such as partial least squares regression (PLSR)~\cite{couture_2013_spectroscopic,serbin_2012_spectroscopic,serbin_spectroscopic_2014,singh_imaging_2015} and wavelet transforms~\cite{banskota_2013_utlity,blackburn_2008_retrieval,cheng_2010_continuous}.
However, these approaches can have important limitations depending on the application. 
First, the empirical nature of these methods can result in sensor, site, and/or vegetation specific relationships, as evidenced by the substantial variability in coefficients and choice of wavelengths across studies~\cite{croft_2014_applicability,huete_2002_overview,knyazikhin_1998_estimation,leprieur_1994_evaluation,liu_2012_assessment,myneni_2002_global,wessels_2012_limits}.
Second, empirical approaches are not a direct mechanistic relationship between spectra and plant properties and therefore do not provide the true connections between optical properties and variables of interest~\cite{knyazikhin_2012_hyperspectral}.
As a result, extrapolating empirical approaches and relationships to larger regions or new locations can be challenging.
Moreover, the indirect, derived data products that arise from such analyses may have a limited capacity to inform ecosystem models~\cite{quaife_2008_assimilating}, as they often introduce assumptions that conflict with the internal logic of the processes represented in these models. 

In contrast, radiative transfer models (RTMs), which provide a more mechanistic link between plant traits and spectral signatures, can be a useful alternative to empirical approaches.
A variety of standalone RTMs exist from the leaf~\cite{dawson_1998_liberty,feret_2008_prospect,ganapol_1998} to canopy scales~\cite{jacquemoud_2009_prosail,kuusk_2001_journal,verhoef_1984_sail,wang_2013_canopy}.
In addition, RTMs are often an important component of dynamic vegetation models, where they are used to calculate surface energy balance and light availability for photosynthesis~\cite{medvigy_2009_mechanistic,nimeister_2010_clumped,kobayashi_2012_modeling}.
In this study, we focus on the leaf-level PROSPECT model~\cite{jacquemoud_1990_prospect,feret_2008_prospect}, which has been extensively used in forward (simulation) mode to develop and test new remote sensing techniques~\cite{croft_2014_applicability,feret_2011_optimizing,lemaire_2004_towards,zarco_tejada_2013_estimating}
as well as to estimate leaf traits from spectral observations via inversion~\cite{atzberger_2012_spatially,feret_2008_prospect,jacquemoud_1995_extraction,jacquemoud_2009_prosail,li_2013_retrieval,li_2011_retrieval,zarco_tejada_2004_needle}.
However, the commonly used approaches for RTM inversion—such as least-squares minimization and look-up tables—fail to directly quantify the uncertainties and account for the correlations among the resulting parameter estimates.
The characterization of uncertainty is a fundamental requirement for drawing meaningful scientific conclusions from results and for assimilating results into statistical or mechanistic models~\cite{cressie_accounting_2009,quaife_2008_assimilating}.

Applying Bayesian statistics to RTM inversion activities provides a direct means to quantify the uncertainty and covariance of parameter estimates while combining multiple sources of information.
The use of independent prior information has been a critical component of RTM inversion as a way to solve the otherwise underdetermined problem of estimating a large number of RTM parameters from a small number of observations%
\cite{combal_2003_retrieval,lauvernet_2008_multitemporal,yao_2008_lai,pinty_2011_exploiting,laurent_2014_bayesian,mousivand_2015_multitemporal}.
While these studies either neglect parameter uncertainty or estimate it using computationally-efficient approximations (e.g. Gaussian posterior distributions), recent work has demonstrated the efficacy of fully-Bayesian Markov Chain Monte Carlo (MCMC) approaches for inversion of the PROSAIL canopy RTM using MODIS (and “MODIS-like”) data%
\cite{zhang_2005_estimating,zhang_2006_characterization,zhang_2009_satellite}.
However, to the authors’ knowledge, such approaches have yet to be applied to hyperspectral data, neither at the canopy nor the leaf scales.
A recent study by Lepine et al.~(2016) \nocite{lepine_2016_examining} further demonstrated that PLSR estimates of canopy nitrogen are less sensitive to spectral resolution than spatial resolution and sensor fidelity, but no comparable analyses has been attempted for other foliar constituents, nor, for that matter, using a physically-based RTM rather than an empirical regression. 
In this study, we examine the effects of measurement spectral characteristics on accuracy, uncertainty, and covariance of leaf traits estimated from spectral inversion of a leaf RTM\@.
First, we demonstrate the applicability of a fully Bayesian approach to leaf RTM inversion and validate this approach using data from the NASA Forest Functional Types (FFT) database of field spectra~\cite{serbin_spectroscopic_2014,singh_imaging_2015}.
Second, we simulate reflectance observations using the spectral response functions of ten common remote sensing platforms and test the accuracy and precision with which our inversion algorithm can retrieve parameters from these observations.
Although such an experiment is highly idealized, it does provide insight on the absolute theoretical limits of RTM inversion by different remote sensing platforms and illustrates how subtle changes in spectral measurement characteristics can affect inversion results. 
More broadly, this work reiterates the power of a Bayesian framework for fully utilizing the vast archive of remote sensing and field spectral observations to enhance our understanding of ecosystem processes.
