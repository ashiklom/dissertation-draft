\section{Conclusions}\label{sec:pecanrtm-conclusions}

This study introduces a novel application of Bayesian spectral inversion to the PROSPECT 5 leaf RTM that explicitly takes into account uncertainty and correlation in parameter estimates.
Validation of our algorithm on a coupled leaf spectral-trait database revealed accuracy comparable to previous inversion algorithms despite only using reflectance observations and the default PROSPECT model (i.e.\ no additional refinement of the specific absorption features).
By simulating reflectance measurements with the spectral characteristics of different remote sensing platforms, we were able to quantify the relationship between spectral resolution and parameter uncertainty.
Although our simulated observations are highly idealized, we believe the resulting patterns in retrieved parameter accuracy and precision are representative of the advantages and limitations of the spectral configurations of different sensors for remote sensing of vegetation.
Our work reinforces the notion that Bayesian spectral inversion provides a powerful and versatile framework for future RTM development and single- and multi-instrumental remote sensing of vegetation, and we encourage members of the remote sensing community to apply and build upon the tools we have developed.
