\chapter{Introduction}
\label{chapter:Introduction}
\thispagestyle{myheadings}

Vegetation is a critical component of the biosphere that mediates climate, modifies the hydrologic cycle, is a haven for biodiversity, and provides many natural resources~\cite{bonan_forests_2008}.
Unfortunately, current predictions about the future of the terrestrial biosphere are highly uncertain, as evidenced by the huge disparities in model projections of the magnitude and direction of the land carbon sink \cite{friedlingstein_uncertainties_2014}.

(MORE INTRODUCTION...)

Plant functional traits are defined as measurable characteristics of plants related to their morphology, physiology, and/or phenology \cite{violle_2007_concept}.

In the first chapter of my dissertation, I evaluate the information content in a well-known global plant trait database (TRY).
In particular, I explore the additional information content in trait correlations and explore the dependence of these correlations on ecological scale.
In the second chapter, I develop methods for obtaining new trait measurements from leaf reflectance spectra through Bayesian radiative transfer model inversion.
In the third chapter, I apply these methods to a large database of leaf spectra to explore the drivers of leaf trait variability, with a particular focus on intraspecific variability caused by environmental stress.
In my final chapter, I validate an ecosystem model by comparing its estimates of surface reflectance through time against satellite surface reflectance measurements.
