\section{Conclusions}

The objective of this study was to calibrate the canopy radiative transfer model inside of the ED2 dynamic demographic vegetation model by comparing its predictions of surface reflectance against direct observations thereof by airborne imaging spectroscopy.
In general, the calibration successfully constrained the posterior distributions of model parameters related to canopy structure (leaf angle, canopy clumping, and leaf area index) for five plant functional types characteristic of temperate forests of the northeastern United States.  
However, comparisons of predicted spectra post-calibration against observations reveal widespread biases, which indicates that there are structural issues with the ED2 radiative transfer model that inhibit its ability to accurately predict surface optical properties.
Sensitivity analyses, along with comparison against an alternative canopy radiative transfer model more commonly used by the remote sensing community (4SAIL), shed additional light on the problem and provides avenues for future exploration and model improvement.
One issue was unrealistically high sensitivity to wood reflectance, which contributed to a positive bias.
Future work could add parameters related to wood to the calibration step, and consider alternative representations of wood reflectance.
This is low-hanging fruit and should be the first refinement to this analysis.
However, wood reflectance alone was insufficient to explain the positive bias in ED2 compared to 4SAIL.
Another source of error suggested in the discussion was excessive sensitivity to soil reflectance.
Future investigation should compare the sensitivities of EDR and 4SAIL to soil reflectance, particularly under dense, closed canopies, to determine whether there is in fact excessive sensitivity.
Even if the sensitivity is correct, this study's use of a fixed soil background likely contributed significant errors for the many sites with young, open canopies and exposed understories, and future work would do well to incorporate soil variability to some extent to the calibration.
Finally, potential utility of doing all this with time series...
